%
%	chap2.tex
%
\chapter{関連研究}
最初にIQに関連する研究について述べ,次にタグの2段階比較に関する
研究について述べる.
\section{IQについての研究}
Palacharlaらは,命令発行幅とIQのサイズを変化させた時の,ウェイクアップ論理と選択論理の遅延を評価した
\cite{Palacharla1997}.また,遅延を小さくするために,IQを複数の
FIFOバッファで構成し,依存する命令を同じFIFOバッファに割り当てる依存ベースのIQを提
案した.この手法では,各バッファの先頭の命令のみ発行可能かチェック
すれば良いので,回路が単純化され遅延が減少する.

Starkらは,IPCをほとんど低下させずに,ウェイクアップ論理と選択論理をパイ
プライン化する手法を提案した\cite{Stark2000}.この手法では,投機的にウェイクアップを行うことで,依存する命令を連続するサイクルで発行できるようにした.

Michaudらは,IQをシフト・レジスタとCAMで構成された発行バッファに分けることで,
複雑さを低減する手法を提案した\cite{Michaud2001}.この手法では,デコード・
ステージで命令
の発行タイミングを予測し,発行までの遅延に応じてシフト・レジスタに挿入する.
そして,シフト・レジスタより順に発行バッファに命令が送られ,発行バッファよ
り命令は発行される.予測が十分正確であれば,発行バッファのサイズは発行幅まで近づけ
られる.

Folegnaniらは,空のエントリの比較器や既にレディなオペランドを持つ比較器
など,タグを比較する必要がない比較器を動作させないことで,消費エネルギー
を削減する手法を提案した\cite{folegnani2001}.

Ponomarevらは,リソース要求に応じてIQのサイズをリサイズすることで,消費
エネルギーを削減する手法を提案した\cite{ponomarev2001}.

Ernstらは,IQに入ってくる命令のうちのほとんどが,はじめから少な
くとも1つのソース・オペランドがレディであると指摘した\cite{ernst2002}.
そしてIQに,2つのソース・オペランドを保持できるエントリに加えて,
1つのソース・オペランドのみ保持できるエントリと,ソース・オペランドを保
持しないエントリを用意し,レディでないソース・オペラ
ンドの数に応じていずれかにディスパッチする手法を提案した.さらにこの手法を実現する
ために,命令の2つのオペランドの内,あとにレ
ディになるオペランドを予測する手法も提案した.

五島らは,ウェイクアップ論理を従来のCAMではなく,依存行列と呼ぶRAMで構成
する手法を提案した\cite{goshima2001}.これによって比較器を用いずに依存す
る命令をウェイクアップすることが可能で,ウェイクアップの遅延を短縮できる.

Brownらは,発行する命令の選択を省略したIQを提案した\cite{brown2001}.これは,命令
がウェイ
クアップされたら,すべて投機的に即時発行する.発行された命令が実際に選
択されたかどうかは後で検証する.これにより,ウェイクアップ
論理と選択論理からなるクリティカルなループから選択論理が排除され,遅延を
短縮できる.

Sassoneらは,依存行列の遅延と電力をより小さくするため
の手法を提案した\cite{sassone2007}.具体的には,従来はすべての命令について,その古さを完全に追跡していたのに対し
て,命令をグループ化してグループ単位で古いものを選択する.これにより,性能低
下を最小限に抑えながら,回路の規模を小さくできる.

Lebeckらは.キャッシュ・ミスするロードのような長いレイテンシの命令に依存する命令を,
IQとは別の待機用バッファに入れ,その長いレイテンシの処理が完了す
るまでIQに挿入しないという方式を提案した\cite{Lebeck2002}.これ
によって,IQが待機する命令で埋ることによって起こるストールの頻度が減り,性能が向上する.

Raaschらは,IQをいくつかのセグメントに分割する方式を提案した\cite{Raasch2002}.こ
の方式では,各命令の依存命令チェーンのレイテンシを元に割り当てるセグメントが決
定される.そして,発行可能になる直前に最下位セグメントである発行バッファに命令を移動する.
この発行バッファでのみ発行を行うことで,すべてのエントリから発行できる通
常のIQと比較して遅延を短縮できる.

Brekelbaumらは,大きな低速のキューと小さいな高速のキューを使い分ける手法を
提案した\cite{brekelbaum2002}.大きな低速のキューにはレイテンシが性能にあまり影響しない命令を
入れる.低速のキューであるタイミングまでにレディにならなかった命令は,クリティカル命令として小さなキューに移される.
小さなキューに入れられたクリティカルな命令は,高速に発行される.

Kimらは,レイテンシが互いに1サイクルの依存関係のある2つの命令をグループ
化し,1つの命令としてIQのエントリでスケジューリングすることで,
依存グラフのエッジのレイテンシ短縮とキューの容量効率を上げる手法を提案し
た\cite{Kim2003}.

Gibsonらは,依存する命令をポインタでつなぎ,ポインタをたどることでウェイ
クアップを行う手法を提案した\cite{Gibson2010}.この方式によりCAMが不要に
なり,電力を削減できる.

Homayounらは,キャッシュ・ミスの処理中に発行幅を半減させることで,IQの消費電力を削減
する手法を提案した\cite{H.Homayoun2011}.発行幅半減中に元の発行幅の半分
以上の命令が発行される場合,一時的にその命令を小さなバッファに移動させることで対応している.

\section{タグの2段階比較についての研究}
CAMによる2段階比較を用いた研究は,数多く存在する
\cite{Zukowski-ISCAS1997}\cite{Pagiamtzis-CAMsurvey2006}\cite{zhang2004}.しかし,発行
キューのウェイクアップ論理のタグ比較に2段階比較を用いる手法は,先行研究
である小林らの研究\cite{kobayashi-thesis}以外では存在しない.小林ら
の研究を元にしたタグ2段階比較については,4章で説明する.
%
%	End of chap2.tex
%
