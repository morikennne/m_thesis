
\chapter{タグ比較回数の概算式の導出}
\label{sec:appendix1}

タグ比較回数の概算式の導出を行う.サブ・セグメントを使用し,また,(R,NR)の命令に対してはスワップを行うとする.

導出にあたり,以下の 2 つの条件を仮定する.
\begin{enumerate}
  \item 命令のソース・タグは 50\% の確率でディスパッチ時にレディであるとする.つまり,(NR,NR),(R,NR),(NR,R),(R,R) である命令の数は全て等しいとする.
  \item 提案手法による IQ の容量効率の低下は全く生じないとする.
\end{enumerate}

ウェイクアップ時のタグ比較は,次の 4 種類に分類することができる.
\begin{itemize}
  \item \ctext{1}(NR,R)である命令の第 1 ソース・タグの比較
  \item \ctext{2}(R,NR)である命令の第 2 ソース・タグの比較
  \item \ctext{3}(NR,NR)である命令の第 1 ソース・タグの比較
  \item \ctext{4}(NR,NR)である命令の第 2 ソース・タグの比較
\end{itemize}

4 種類のタグ比較回数がセグメント化によってどの程度削減できるかを考える.メイン・セグメント数を M-seg,サブ・セグメント数を S-seg と表記する.

\ctext{1}の場合,第 1 ソース・タグが IQ の第 1 ソース・タグのフィールドに書き込まれるため,タグ比較回数は 1/M-seg となる.\ctext{2}の場合は,スワップを行い,第 2 ソース・タグが IQ の第 1 ソース・タグのフィールドに書き込まれる.したがって,\ctext{1}の場合と同様にタグ比較回数は 1/M-seg となる.

\ctext{3}の場合,第 1 ソース・タグはIQ の第 1 ソース・タグのフィールドに書き込まれるため,タグ比較回数は 1/M-seg となる.\ctext{4}においては,第 2 ソース・タグは IQ の第 2 ソース・タグのフィールドに書き込まれるため,タグ比較回数は 1/S-seg となる.

仮定 1 より,\ctext{1}〜\ctext{4}の 4 種類のタグ比較の回数は等しい.つまり,\ctext{1}〜\ctext{4}のタグ比較回数は,全体の 25\%(1/4)である.したがって,BASE モデルに対する提案手法の相対タグ比較回数は,以下の式で概算ができる.

\[
  \frac{1}{4}\times\frac{1}{M\_seg} + \frac{1}{4}\times\frac{1}{M\_seg} + \frac{1}{4}\times\frac{1}{M\_seg} + \frac{1}{4}\times\frac{1}{S\_seg}
  = \frac{3}{4}\times\frac{1}{M\_seg} + \frac{1}{4}\times\frac{1}{S\_seg}
\]

なお,サブ・セグメントを使用しない場合は,\ctext{4}のタグ比較回数が削減されず(1/1),残りのタグ比較に関しては全て1/セグメント数だけ削減される.したがって,上記の式ににおいて$M\_seg$ をセグメント数とし,$S\_seg=1$ として計算すれば良い.