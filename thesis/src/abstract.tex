
\chapter*{概要}
\markboth{概要}{概要}
発行キューは電力密度の大きいホット・スポットとして知られている.ホット・スポットは,デバイスの摩耗故障を引き起こし,誤動作やタイミング・エラーを引き起こす.発行キューが大きな電力を消費する原因は,ウェイクアップ論理のタグ比較回路にある.この回路はCAMで構成されており,全てのデスティネーション・タグと発行キュー内の全てのソース・タグとの多数の比較を一斉に行うため,非常に大きな電力を消費する.そこで本論文では,大容量 CAM の研究分野で提案されている手法を応用し,タグ比較による消費電力を削減する手法を提案する.本手法では,発行キューを複数のセグメントに分割する.命令は,ソース・タグの下位ビットがセグメント番号と一致するセグメントにディスパッチする.そして,ウェイクアップ時には,デスティネーション・タグの下位ビットが一致するセグメントにあるタグ比較器のみを動作させる.一致しないセグメントの比較器は動作しないため,タグ比較器の動作回数を削減できる.

本手法では,命令がディスパッチされるセグメントに空きがない場合,他のセグメントに空きがあってもディスパッチできないためストールする.この結果,発行キューの容量効率が低下するという問題が生じる.この問題は,発行キューの容量効率が重要なプログラムにおいて性能低下を引き起こす.そこで本論文では,容量効率を重視したディスパッチ・アルゴリズムと,タグ比較の積極的な削減を重視したディスパッチ・アルゴリズムを動的に切り替える手法を提案する.本手法は,発行キューの容量効率が重要な場合は容量効率の低下による性能低下を抑制し,そうでない場合は積極的にタグ比較器の動作回数を削減することを可能とする.提案手法を SPEC CPU 2017 を用いて評価を行った.結果,性能低下を最大で 5\% 以下に抑えつつ,タグ比較器の動作回数を平均で 82\% 削減できることを確認した.
