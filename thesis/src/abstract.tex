
\chapter*{概要}
\markboth{概要}{概要}
近年のサーバー・アプリケーションや,JavaScriptで書かれたWebアプリケーションでは,従来のアプリケーションと比べて命令キャッシュ・ミスが特に多く発生することが知られている.これに対し,命令キャッシュ向けのプリフェッチャが多く研究されており,非常に高いキャッシュ・ヒット率が達成されている.しかし,それらのプリフェッチャでは高い性能をもつものほど大きな追加資源が必要になる.例えば,最先端の命令プリフェッチャであるProactive Instruction Fetchでは,L1命令キャッシュそのものより大きなテーブルを必要とする.また,プリフェッチのミスに対するカバー率こそ高いものの,無駄なプリフェッチを多く実行してしまい,電力を無駄に消費してしまう場合がある.

これに対し,本論文では命令プリフェッチのアプローチではなく,フェッチ・ステージのパイプライン構造の工夫によりフェッチ・スループットを向上させる手法を提案する.本提案手法はプリフェッチャと異なり,複雑な機構や大きなテーブルも必要なく,無駄なメモリアクセスを全く行わない.さらに,性能低下のデメリットなしに命令キャッシュ・ミスによるストールを削減し,フェッチ・スループットを向上させることができる.
提案手法をサーバー向けのベンチマークを用いて評価したところ,プリフェッチを行わない場合と比較して最大 25.8\%,平均で13.0\%の性能向上を達成した.また,最先端の命令プリフェッチャと比較して平均4.8\%の性能向上が得られることを確認した.
